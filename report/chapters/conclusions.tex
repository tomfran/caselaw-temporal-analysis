\section{Concluding remarks}
As showed in the previous subsection, it is possible to explore
in various ways the dataset by exploiting different techniques, such as topic modelling, word embeddings, semantic
shifts and words frequencies.

While some findings have been reported, and it is clearly possible and easy to think and search about other ideas,
it is difficult to evaluate in a theoretical way the results of the analysis. In the literature about
legal text analysis, almost all methods consists of classification which rely on supervised methods, 
especially related to \emph{querying systems}, exploiting the manual help of domain experts in building the dataset 
and evaluating the results.~\cite{caselaw_query}

We decided to focus on an unsupervised methodology, trying to explore in a broad manner the dataset in order to
extract as much latent information as possible. This work could serve as a starting point to expand what has been
discovered, by taking advantage of the already implemented techniques such as topic modelling but considering new
features of interest, or by implementing supervised techniques.

Some hints of possible future works could be the exploitation of the geographical areas of the judgments,
or the \emph{type of
opinions}, such as dissenting and consenting, or even considering the individual judges in order to specialize
the work done with topic modelling and word embeddings. It could be also possible to 
implement new supervised techniques,
which could help to predict court judgments based on certain words and contexts.~\cite{argumentation_mining}