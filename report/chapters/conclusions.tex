\section{Concluding remarks}
As showed in the Interesting findings subsection, it is possible to explore
in various interesting ways the analysis by exploiting various techniques: topic modelling, word embeddings, semantic
shifts, words frequencies\ldots

While it many interesting findings have been showed, and it is possible and easy to search for other ideas, it has
been difficult to evaluate in an empirical way the results of the analysis; much literature has been explored on the
legal text analysis, but almost all of it consists in classification methods which rely on supervised methods
(especially related to querying systems \cite{caselaw_query}),
exploiting the manual help of domain experts in building the dataset and evaluating the results.

We decided to focus on a more unsupervised methodology, trying to explore in a broad manner the dataset in order to
extract as much hidden info as possible. This paper/work could serve as a starting point to expand what has been
discovered, by taking advantage of the already implemented techniques such as topic modelling but considering new
features of interest, or by implementing supervised techniques.

Some hints of possible future works could be the exploitation of the geographical areas of the judgments, the type of
opinion (dissenting, consenting\ldots~\cite{argumentation_mining}) or even the individual judges to specialize the work done with topic modelling
and word embeddings, possibly implementing new supervised techniques which could help to predict court judgments
based on certain words and contexts.