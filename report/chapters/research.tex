\section{Research question and methodology}

The goal of the project is to analyze, from a collection 
of court decisions, the relevance of terms in time while 
also gathering information about possible correlations between 
them. The reasons to perform such a study are various, firstly, 
one could find interesting and unexpected correlations between terms, 
also, studying them in time could reveal a change of 
context of some words.\\
The research question is focused in particular on three main areas of interest:
narcotics, weapons and investigations. Thus, words related to this three main sets
are the most interesting ones to analyze.

The \emph{Mining the Harvard Caselaw Access Project} paper has been used as a starting point for
both the analysis and the visualization part, which explores a different area of interest but with some
common ideas.~\cite{harvard_caselaw}

\subsection{Initial idea}
\label{res-med}
In the first place, the methodology aimed to divide the dataset into the three main areas, 
corresponding to the area of interests previously introduced. 
Having documents divided in areas, would have made possible to perform topic modelling and language modelling
on the three areas, possibly considering the temporal factor in the process.

This approach would have allowed to directly confront the three sets also in a temporal way, showing differences in
the language and topics between words regarding drugs, narcotics and investigations respectively.

\noindent Some scientific literature has been explored and tested, based on the idea of \emph{Guided topic modelling} 
but it has been soon discovered that the dataset comprehended many other topics other than the
three main areas of interest.~\cite{glda} Another drawback was the fact that even when finding promising results 
with the guided method, the three topics were somehow intersecting too much, probably because the 
area of interests are closely related to each other in real life.

\subsection{Refined methodology}

After coming to a dead end with the approach presented in the previous Subsection, we opted for a different
overall methodology, shifting from guided topic modelling towards a classical one and performing the temporal
analysis with an heavy use of word embeddings.